% the abstract

The most recent advances in the techniques to observe and measure biological processes are characterised by the generation of large amounts of data. The more data we gather the better the chances to find clues for understanding the systems of life. This however, is only true if the methods to analyse the generated data are efficient, effective and robust enough to overcome the challenges intrinsic to the management of big data.

The computational tools required to endeavour these challenges should also consider the requirements of today's research: science requires very specialised knowledge to understand the particularities of each object of study; and simultaneously, it is seldom possible to explain them without considering their relationship with other processes, entities or systems.

This thesis explores two closely related fields: integration and visualisation of biological data. Which we believe are fundamental to the creation of scientific software tools that respond to the exposed needs.

The distributed annotation system (DAS), is a community project that supports the integration of data from federated sources and its visualisation on web and stand-alone clients. We have extended the DAS protocol to improve its search capabilities and also to support feature annotation by the community. We have also collaborated on the implementation of MyDAS, a server to facilitate the publication of biological data following the DAS protocol, and contributed in the design of the protein DAS client called DASty. We have also developed probeSearcher, a tool that uses the DAS technology in order to facilitate the identification microarray chips that includes probes mapping proteins of interest.

We participated in the formation of BioJS, an open source library of visualisation components for biological data. This thesis includes a description of the project, our contributions to it and some developed components that are part of it.

PINV is a web based visualiser of protein-protein interaction (PPI) networks, that takes advantage of the features of modern web technologies in order to explore PPI datasets on an almost ubiquitous platform (the web) and facilitates the collaboration between scientific peers. This thesis includes the description of the design and development processes of PINV, as well as current use cases that have benefit from the tool and which feedback has been the source of several improvements in PINV.

We sincerely hope that  the tools and developments presented in this thesis contribute to the elucidation of biological knowledge.