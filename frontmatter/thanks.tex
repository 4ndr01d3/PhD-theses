% the acknowledgments section

\newthought{What a hike!}

I'm ready for this, it is going to be a short and pleasant walk, with nice views and beautiful landscapes, what can go wrong? if with the hike I just finished I pretty much have come half way, the rest should be easy, no surprises!

It has been a long time and Im still on my way, but I'm not complaining, why would've I? Besides, I wasn't that wrong. Yes, it wasn't short, and yes there were (are) scary moments, and yes, not everything was pretty, but you have to go through repetitive dusty hills if you want to get to the viewpoint; and at the end, it is only possible to appreciate the view, when you recognise the effort that you have put into it, that feeling of achievement is mine, no one can't take it away. 

But clearly I wouldn't be here if I was by myself. The path wasn't always clear and there were times that I felt lost, how many times I have to stop for directions? way to many! And no one like Nicky to calmly show me that I wasn't lost, that there was a path in front of me, that maybe I should just have some water, breath and keep walking. 

Sometimes I ask myself, how I end up committing to this challenge, and then realise that the question is not how, but who motivate me to do it, and here the list is long, but in the top of it is my family, because they teach me to walk, sure they also gave me the boots and the raincoat, but that is worth nothing if you can't walk. My mom who worries about a thousand details that I didn't consider when I gave the first steps, and my dad who would give everything away, to get a chopper if I need to be rescued. How can I consider myself brave about coming here, if I'm a phone call away of a rescue chopper? And my brother, oh Juancho, how I wish you were here, running over the same old ground.

I knew this was a technical hike, but I'm lucky enough to know some people who has walk similar ones, and their invaluable advise were always there. Starting from Rafa, who gave me a place to stay in the first camp I found, showing me around, and allowing me to safely explore during those cold times, he also introduced me to Henning, who for a short period guided me and allowed me to be part of his group, in which I have the pleasure of met so many professional hikers, from whom I hope I have learned one or two tricks.

But you can't walk all the time, and there were quite a few friends that i met in the cold camp, with who I played, sang and danced. Very good memories from Ruben, Claudia, Jose, Bernat, Leyla, Fernanda, Ana, Josue and Jorge.

Back in the base camp I met who have become my partner in crime, the person that walked by my side ever since, the one would take care of me if I twist an ankle, even if she is upset with me. Thanks Julie, thanks mi nena, I really hope this is only the first of many adventures together.

It is like scoring a goal with the defenders on you and the goalie centimetres away of the ball, and you are calm enough to hit the ball, gently but firmly, strong enough to elevate it over the body of the keeper, but soft enough to make it fall in the middle of the three posts. Goal!


\emph{ Manuel CBIO Pineapples }