% the acknowledgments section

\newthought{What a hike!}

\vspace{3mm}

I'm ready for this, it is going to be a short and pleasant walk, with nice views and beautiful landscapes, what can go wrong? I pretty much have come half way with the hike I just finished, the rest should be easy, no surprises!

A lot of time has passed, and I'm still on my way. But I'm not complaining, why would I? Besides, I wasn't that wrong. Yes, it wasn't short, and yes there were (are) scary moments, and yes, not everything was pretty, but you have to go through repetitive dusty hills if you want to get to the viewpoint; and at the end, it is only possible to appreciate the view, when you recognise the effort that you have put in, that feeling of achievement is mine, no one can take it away. 

But clearly I wouldn't be here if I was by myself. The path wasn't always clear and there were times that I felt lost, how many times have I had to stop for directions? Far too many! And there is no one like Nicky to calmly show me that I wasn't that lost, that there was a path in front of me, that maybe I should just have some water, breathe, and keep walking. Thanks for your help and support Nicky.

A lesson learnt is the importance of a good base camp, a warm place where you can rest after an extraneous day, and if you are lucky enough you might find friends in those sort of places, the kind of friends who give you a ladder to brake into your own house, or talk to you, even when they barely understand you. Thanks Kate, Louis, Becca, Michelle and Richard.

\vspace{3mm}

Sometimes I ask myself, how I ended up committing to this challenge, and then realise that the question is not how, but who motivated me to do it, and here the list is long. But right at the top of it, is my family, because they tough me to walk, sure they also gave me the boots and the raincoat, but that is worth nothing if I didn't know how to walk. My mom who worries about a thousand details that I didn't consider when I gave the first steps, and my dad who would give everything away, to get a chopper if I need to be rescued. How can I consider myself brave about coming here if I'm a phone call away of a rescue chopper? And my brother, oh Juancho, how I wish you were here, running over the same old ground. Gracias M\'a! Gracias P\'a! Gracias Juancho!

\vspace{3mm}

I knew this was a technical hike, but I'm lucky enough to know some people who have walked similar ones, and their invaluable advice was always there. Starting from Rafa, who gave me a place to stay in the first camp I found, showed me around, and allowed me to safely explore during those cold times. He also introduced me to Henning, who for a short period guided me and allowed me to be part of his group, in which I have had the pleasure of meeting so many professional hikers from whom I hope I have learned one or two tricks.

But you can't walk all the time, and there were quite a few friends that i met in that cold camp, with whom I played, sang and danced. Very good memories from Ruben, Claudia, Jose, Bernat, Leyla, Fernanda, Ana, Josue and Jorge.

Back in the base camp I met the person who has become my partner in crime, the person that walked by my side ever since, the one who takes care of me if I twist an ankle, even if I have made her upset. She has read this whole story and has given me lots of advices (sic) to make it better, so, if this whole thing is readable it is only because of her. Thanks Julie, thanks mi nena, I really hope this is only the first of many adventures together.

\vspace{3mm}

There were hills bigger than my skills, but then I was lucky enough to realise that, by collaborating, the obstacles are easy to deal with - thanks again to Rafa, but also to Manuel, Seb, Alex, Leyla, Jose, Andy, Jon and all the people of the BioJS and DAS groups. But the group that gave me the most support during this time was CBIO, in which I always found people willing to help me, with whom I shared experiences, making my trip easier, and hopefully I made theirs better. We might not all been walking to the same peak, but we definitely share quite a few paths. Thanks Ayton, Holy, Gaston, Kenneth, Jon, Renaud, Rebone and everyone in the group.

So many memories build up during this campaign, for example to find a couple of waterfalls on side paths, or to be able to stop and have a pineapple on the road, and sharing it knowing that our friends really appreciate a good pineapple, wouldn't you? Nats, Jon, Danie, Brett, Tamsin, Sean, Sarah, Ayton and Julie. You guys rock, by the way.

\vspace{3mm}
And now here at top of this hill, I can only be grateful, it is like scoring a goal with the defenders on you and the goalie centimetres away of the ball, and you are calm enough to hit the ball, gently but firmly, strong enough to elevate it over the body of the keeper, but soft enough to make it fall in the middle of the three posts. It is yours, yeah! but who made the pass?, and who stole the ball in the first place? and who were the warriors defending the whole time?, and who made the fancy moves just for the show? Simply, your friends, your team. \emph{Goal!}
