\begin{savequote}[75mm] 
La verdad no penetra en un entendimiento rebelde. Si todos los lugares de la tierra est\'{a}n en el Aleph, ah\'{i} estar\'{a}n todas las luminarias, todas las l\'{a}mparas, todos los veneros de luz.
\qauthor{Jorge Luis Borges, El Aleph} 
\end{savequote}

\chapter{Integration of Information in Bioinformatics}

\newthought{The data analysed in bioinformatics comes from diverse and heterogeneous sources}, for example, the data might be captured from wet-lab experiments or deduced with \emph{in-silico} procedures, it can refer to nucleic information and sequenced data, but also to expression levels and other protein information, it is possible to analyse isolated organisms or to gather information from multiple species. It all depends on the purpose of the research and the availability of data, however it is almost inevitable to have to integrate more than one of this sources in order to tackle todays research challenges.

Immediately after this, we will discussing MyDAS, a server application for federated data that was updated as part of this PhD project as one of the contributions to the integration of data in bioinformatics. We will be discussing a 

\begin{description}
	\item[First author publications]:\\
		\begin{enumerate}
			\item Gustavo A. Salazar et al. \emph{MyDas, an Extensible Java DAS Server} \emph{PLoS ONE} 2012, 7(9): e44180. doi: 10.1371/journal.pone.0044180
			\item Gustavo A. Salazar et al. \emph{DAS Writeback: A Collaborative Annotation System} In \emph{BMC Bioinformatics} 2011, 12:143  doi:10.1186/1471-2105-12-143
		\end{enumerate}
 	\item[Coauthor publications]:\\
		\begin{enumerate}
			\setcounter{enumi}{2}
			\item Jose M. Villaveces et al. \emph{Dasty3, a WEB framework for DAS} in  \emph{Bioinformatics}  2011, 27 (18): 2616-2617.
			\item Rafael C. Jimenez et al. \emph{myKaryoView: A Light-Weight Client for Visualization of Genomic Data. } In \emph{PLoS ONE} 2011, 6(10): e26345. doi: 10.1371/journal.pone.0026345
		\end{enumerate}

	\item[Author's Contibutions]:\\
		\begin{enumerate}
			\item Conceived and designed the experiments: GS. Performed the experiments: GS AJ. Wrote the paper: GA LG PJ RJ. Critical revision of the manuscript for important intellectual input: RJ AQ AJ NM MM SH HH. Technical and material support: AJ NM MM SH HH. Supervision: NM MM SH HH. Study concept: GS LG PJ AQ RJ HH. Architectural design: PJ GS. Software development: GS LG PJ AQ. Evaluation of the compatibility with DAS protocol: AJ.
			\item Critical revision of the manuscript for important intellectual input: RJ, AG, HH, NM and EB. Technical and material support: HH, NM and EB. Study supervision: HH, NM and EB. Study concept: GS, RJ and AG. Architectural design: GS and EB. Software development: GS. Drafting of the manuscript: GS. Design of the usability experiment: GS, NM and EB. All authors read and approved the final manuscript.
			\item Critical revision of the manuscript for important intellectual input: JV, RJ LG,GS, BG, NM, MM, AG and HH. Technical and material support: HH, NM, AG and MM. Study supervision: HH, NM, AG and MM. Study concept:  RJ and HH. Architectural design: JV, GS, BG and JG. Software development: JV. Drafting of the manuscript: JV, AG and LG. All authors read and approved the final manuscript.
			\item Conceived and designed the experiments: RCJ MC NM JD. Performed the experiments: RCJ MC. Analyzed the data: MC. Contributed reagents/materials/analysis tools: GS BG. Wrote the paper: MC.
		\end{enumerate}
\end{description}


\textbf{}



\section{MyDAS}
\section{DAS Writeback}
\section{DAS Visualizations}
\subsection{Dasty}
\subsection{myKaryoView}
\section{Discussion}