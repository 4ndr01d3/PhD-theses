\begin{savequote}[75mm] 
Our posturings, our imagined self-importance, the delusion that we have some privileged position in the Universe, are challenged by this point of pale light.
\qauthor{Carl Sagan, Pale Blue Dot, 1994} 
\end{savequote}

\chapter{Visualization in Bioinformatics}

\newthought{A single point in a photograph can represent our whole known world}, as shown in the famous image acquired by the Voyager 1 spacecraft in 1990. From that perspective, is impossible to perceive  all details that conform our planet, from rivers to highways, from mountains to buildings, even countries or full continents and oceans are undistinguished in the mentioned picture. Nonetheless, the image gave us a insight of the vastness of the universe in comparison with our known world.

As in the previous example the same object can be seen from different perspectives each of them will highlight some features and hide others, hence the importance of choosing the right representation for the object in display.



\begin{description}
	\item[First author publications]:\\
		\begin{enumerate}
			\item Gustavo A. Salazar et al. \emph{PPI layouts: BioJS components for the display of Protein-Protein Interactions} in  \emph{F1000Research} 2014, 3:50 (doi: 10.12688/f1000research.3-50.v1) [v1; ref status: indexed, http://f1000r.es/2u5]
			\item Gustavo A. Salazar et al. \emph{A web-based protein interaction network visualizer} In \emph{BMC Bioinformatics} 2014, 15:129  doi:10.1186/1471-2105-15-129
		\end{enumerate}

	\item[Coauthor publications]:\\
		\begin{enumerate}
			\setcounter{enumi}{2}
			\item John Gomez et al. \emph{BioJS: an open source JavaScript framework for biological data visualization} in  \emph{Bioinformatics}  2013 29 (8): 1103-1104. doi: 10.1093/bioinformatics/btt100
			\item Manuel Corpas et al. \emph{BioJS: an open source standard for biological visualisation – its status in 2014} In \emph{F1000Research} 2014, 3:55 (doi: 10.12688/f1000research.3-55.v1)  [v1; ref status: indexed, http://f1000r.es/2yy]
		\end{enumerate}
 
	\item[Author's Contibutions]:\\
		\begin{enumerate}
			\item Critical revision of the manuscript for important intellectual input: GS, AM and NM. Supervision: NM. Study concept: GS, AM and NM. Software development: GS. Drafting of the manuscript: GS and AM. All authors have read and approved the final manuscript.
			\item Critical revision of the manuscript for important intellectual input: GS, AM, GM, HR, RA and NM. Study concept: GS, AM and NM. Software Design: GS, AM, GM, HR, RA and NM. Software development: GS and AM. Creation of datasets: GM, HR and RA. Software Testing: GM, HR, RA and NM. Software Documentation: GS and RA. Drafting of the manuscript: GS, HR and NM. Supervision: NM. All authors read and approved the final manuscript;
			\item All authors have participated in the development of the BioJS community through provision of code, meeting attendance or writing of grants.
			\item All authors have participated in the development of the BioJS community through provision of code, meeting attendance or writing of grants.
		\end{enumerate}
\end{description}


%However most of those efforts are tied to specific projects or in the best cases to specific types of data, and interaction between visualization tools abroad several domains is virtually non existing, and often is limited to links between web resources.

%We tackle this issue by proposing a software architecture of multi-domain visualization components that allows interaction between, making possible to analyse data from more than one domain in a single view. The tool that implements such architecture is web based, developed with HTML5 and using BioJS components as the standard base to allow communication between components.

, then BioJS is described in depth, including a set of components that have been developed. A section describing a tool for the visualisation of protein-protein interactions 


\section{BioJS: A JavaScript framework for Biological Web 2.0 Applications }
\subsection{Developed Components}
\subsubsection{Protein-Protein Interactions networks}
\subsubsection{Chromosome Viewer}
\subsubsection{Other Components}

\section{PINV, a web-based Protein Interaction Network Visualiser }
\subsection{Architecture}
\subsection{Implementation}
\subsection{Description of the application}
\subsection{Use Cases}

\section{Spacial clustering for big networks on a force-directed layout}

\section{Discussion}


